\documentclass[11pt]{article}
\usepackage{graphicx}
\usepackage{amsmath}
\usepackage{mathtools}
\usepackage{float}
\usepackage{multirow}
\usepackage[spanish, mexico]{babel}
\usepackage[utf8]{inputenc}
\usepackage[hyphens]{url}
\usepackage[
    colorlinks = true,
    citecolor = red,
    urlcolor = blue]{hyperref}
\usepackage{cite}
\usepackage{listings}
\usepackage{subcaption}
%\usepackage{algorithm}
\usepackage{program}
\usepackage{algorithmic}
%\usepackage[]{algorithm2e}
\usepackage{lipsum}
\usepackage{courier}
\usepackage{xcolor}
\usepackage{float}
\usepackage{textcomp}
\usepackage{fancyvrb}
\usepackage{rotating}
\definecolor{listinggray}{gray}{0.95}
\definecolor{lbcolor}{rgb}{0.95,0.95,0.95}
\lstset{
    backgroundcolor = \color{lbcolor},
    tabsize = 4,
    language = Matlab,
    basicstyle = \scriptsize,
    upquote = true,
    aboveskip = {0.5\baselineskip},
    columns = fixed,
    showstringspaces = false,
    extendedchars = true,
    breaklines = true,
    prebreak = \raisebox{0ex}[0ex][0ex]{\ensuremath{\hookleftarrow}},
    frame = none,
    showtabs = false,
    showspaces = false,
    showstringspaces = false,
    identifierstyle = \ttfamily,
    keywordstyle = \color[rgb]{0,0,1},
    commentstyle = \color[rgb]{0.133,0.545,0.133},
    stringstyle = \color[rgb]{0.627,0.126,0.941},
    basicstyle = \ttfamily\small,
    breaklines = true,
    numbers = left,
    numberstyle = \footnotesize,
    stepnumber = 1,
    numbersep = 0.5cm,
    xleftmargin = 0.2cm,
    xrightmargin = 0.3cm,
    frame = tlbr,
    framesep = 5pt,
    framerule = 0pt,
    literate =
    {ó}{{\'o}}1
    {á}{{\'a}}1
}
\usepackage{parskip}

\setlength{\parindent}{10pt}
\interfootnotelinepenalty = 10000

\title{\textbf{Proyecto:\\
Búsqueda de línea e interpolación cuadrática aplicados a problemas de crecimiento poblacional logístico} \vspace{1cm}}
\author{
    Luis M. Román\\
    117077
    \and
    Omar Trejo Navarro\\
    119711 \\
    \and
    Fernanda Mora Alba\\
    103596  \\
    \vspace{0.5cm}
    \and
    \Large{ITAM} \\ \\
    \large{Análisis Aplicado} \\ \\
    \large{Zeferino Parada} \\ \\
}
\date{\today}

\begin{document}
\newtheorem{theorem}{Theorem}[section]
\VerbatimFootnotes
\maketitle


\begin{abstract}
haremos uso de las

\end{abstract}


\section{Introducción}

El modelo logístico de crecimiento está determinado por la ecuación diferencial


\begin{equation}
\frac{dP}{dt} = rP(t)(1 - \frac{P(t)}{K})
\end{equation}


donde P(t) es la población al tiempo t, r es la tasa de crecimiento y K es una constante con la cantidad máxima permitida de la población.

Ahora bien, para resolver esta ecuación diferencial, notemos primero que podemos separarla de la siguiente manera

\begin{equation*}
\frac{dP}{dt[P(t)(k - P(t))]} = \frac{r}{k}
\end{equation*}


Lo que nos da

\begin{equation*}
\begin{split}
\int\frac{dP}{P(t)(k - P(t))}                     & = \int\frac{r}{k}dt\\
\rightarrow & \\
\int\frac{dP}{kP(t)} + \int\frac{dP}{k(k - P(t))} & = \int\frac{r}{k}dt\\
\rightarrow & \\
\frac{1}{k}[ln(|P(t)|) - ln(|k - P(t)|)]          & = \frac{r}{k}t\\
\rightarrow & \\
ln(\frac{|P(t)|}{|k - P(t)|})                     & = rt + c\\
\rightarrow & \\
|P(t)|                              & = |k - P(t)|e^{rt + c}\\
\end{split}
\end{equation*}

Suponemos $P(t) > 0$ y $k -P(t) >0$ de aquí que

\begin{equation*}
\begin{split}
P(t) & = \frac{ke^{rt + c}}{1+e^{rt + c}}\\
     & = \frac{k}{1+e^{-(rt + c)}}
\end{split}
\end{equation*}


\section{Modelado de ventas de IPAD}

Haremos uso de (1) para modelar el proceso de adaptación de IPAD en la población. Sean $x = (r, K, P_0)^T$ los parámetros que se busca determinar y $(t_k, v_k)$ las ventas observadas en el trimestre k. Si definimos
\begin{centering}
\begin{equation}
r_k(x) = P(t_k) - v_k,\quad k = 1, 2, \dots, 16
\end{equation}
\end{centering}

Entonces buscamos minimizar la función

\begin{centering}
\begin{equation}
f(x) = \frac{1}{2}\displaystyle\sum_{i = 1}^{16}r_k(x)^2
\end{equation}
\end{centering}

Utilizaremos la dirección de Newton y el método de búsqueda de línea con interpolación parabólica para minimizar la función objetivo.

Contamos con los siguientes datos \footnote{
los datos pueden ser obtenidos en \url{http://en.wikipedia.org/wiki/IPad}}

\begin{center}
\begin{tabular}{| l | c |}\hline
Trimestre & Usuarios\\\hline
10Q3 & 3.27  \\
10Q4 & 4.19  \\
11Q1 & 7.33  \\
11Q2 & 4.69  \\
11Q3 & 9.25  \\
11Q4 & 11.12 \\
12Q1 & 15.30 \\
12Q2 & 11.80 \\
12Q3 & 17.00 \\
12Q4 & 14.00 \\
13Q1 & 22.90 \\
13Q2 & 19.50 \\
13Q3 & 14.60 \\
13Q4 & 14.10 \\
14Q1 & 26.00 \\
14Q2 & 16.35 \\\hline
\end{tabular}
\end{center}

\scriptsize{cantidades en millones}

El pseudo código utilizado para resolver el problema es el siguiente

\begin{program}
\mbox{Optimización con dirección de Newton:}
\BEGIN \\ %
Dada f(x) = \frac{1}{2}\displaystyle\sum_{i = 1}^{16}r_k(x)^2 \quad \epsilon \quad y \quad x_0
  \WHILE |\nabla f(x)|_1\geq \epsilon \DO
     |expt|(2,i); \\ |newline|() \OD %
\rcomment{This text will be set flush to the right margin}
\WHERE
\PROC |expt|(x,n) \BODY
          z:=1;
          \DO \IF n=0 \THEN \EXIT \FI;
             \DO \IF |odd|(n) \THEN \EXIT \FI;
\COMMENT{This is a comment statement};
                n:=n/2; x:=x*x \OD;
             \{ n>0 \};
             n:=n-1; z:=z*x \OD;
          |print|(z) \ENDPROC
\END

\end{program}

\section{Agricultura}



\end{document}
